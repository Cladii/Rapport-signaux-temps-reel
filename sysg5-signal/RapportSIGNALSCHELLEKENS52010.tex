\documentclass{article}

\usepackage[a4paper, total={6in, 8.5in}]{geometry}
\usepackage[francais]{babel}
\usepackage{color}
\usepackage[utf8]{inputenc}
\usepackage[T1]{fontenc}
\usepackage{listingsutf8}

\definecolor{greenCode}{rgb}{0, 0.6, 0}	

\lstdefinestyle{code}{
	inputencoding=utf8/latin1,
	language=C,
	numbers=left,
	frame=tb,
	tabsize=4,
	columns=fullflexible,
	frame=single,
	breaklines=true,
	showspaces=false,
	showstringspaces=false,
	commentstyle=\color{red},
	keywordstyle=\color{blue},
	stringstyle=\color{greenCode}
}

\lstdefinestyle{output}{
	inputencoding=utf8/latin1,
	frame=tb,
	tabsize=4,
	columns=fullflexible,
	frame=single,
	breaklines=true,
	showspaces=false,
	showstringspaces=false,
}

\title{Les signaux et les signaux temps-réel}
\author{52010 Schellekens Arnaud}
\date{16 Nomvembre 2020}

\setcounter{tocdepth}{2}

\begin{document}

\pagenumbering{gobble}
\maketitle
\newpage

\tableofcontents
\newpage

\pagenumbering{arabic}

\section{Introduction}
\paragraph{} La gestion des signaux entre processus peut être très utile dans beaucoup de cas concrets. 
Le principe est simple : un processus peut envoyer des signaux à un autre processus ou lui-même et le destinataire doit immédiatement réagir. 
Il peut réagir de différentes façons en ignorant simplement le signal ou en exécutant le traitement associé à celui-ci.
Nous avions déjà vu que le noyau en utilisait pour arrêter notre programme ou encore lorque l'on abordait le fork où le processus fils notifiait le père de 
sa mort.\footnote{\label{note} Nous ne rapellerons pas tout ce qui a déjà été vu au cours.}.

\subsection{Généralités}
\paragraph{} Il existe 32 signaux différents ayant un numéro associé et un nom défini sauf le signal numéro 0 qui est un
peu plus spécial. Ceux-ci se trouvent dans le fichier d'en-tête <signal.h>.
Pour utiliser ces signaux, il faut toujours utiliser le nom symbolique du signal et non son numéro car le numéro d'un 
même signal peut varier en fonction du système ou de la machine.Il existe aussi les signaux temps-réel qui ne sont pas 
supportés sur tous les systèmes. Afin de savoir si la machine supporte les signaux temps-réel, il faut utiliser 
la constante symbolique \_POSIX\_REALTIME\_SIGNALS qui est définie dans $<unistd.h>$. 
Ensuite, nous ne savons pas combien de signaux sont disponibles et nous devons éviter d'utiliser leur numéro pour les utiliser. 
Pour pouvoir y parvenir, nous avons à notre disposition deux constantes symboliques importantes : SIGRTMIN et SIGRTMAX. 
Ils représentent respectivement le numéro du plus petit signal et du plus grand signal temps-réel. 
Grâce à ces deux valeurs, nous pouvons accéder à tous les signaux temps-réel en utilisant les deux bornes. \linebreak
Voici un exemple introduisant l'utilisation des signaux temps-réel :

\subsubsection*{Code}

\lstinputlisting[style = code]{codes/nb_signal.c}

\subsubsection*{Output}

\lstinputlisting[style = output]{outputs/nb_signal.txt}

\paragraph{} Une bonne pratique lorsque nous utilisons des signaux temps-réel est d'utiliser des macros afin de nommer le signal et d'améliorer la lisibilité du code.
Une autre pratique est de vérifier si la machine possède suffisamment de signaux pour notre programme grâce à ses macros que nous avons définies. 

\paragraph{} Nous remarquons aussi que les signaux temps-réel commencent à la valeur 34 et non 32 comme on pourrait s'y attendre. 
Nous pouvons aussi observer que 30 signaux sont à notre disposition au lieu des 32 disponibles sur la machine. 
Lorsque nous affichons la liste complète des signaux disponibles sur la machine, nous voyons le même problème. Où sont passés les signaux 0, 32 et 33?

\lstinputlisting[style = output]{outputs/killlist.txt}

\paragraph{} En lisant la documentation sur les signaux temps-réel, nous pouvons lire que l'implémentation des threads glibc POSIX utilise en interne deux ou trois de ces signaux.
Pour éviter des conflits entre nos processus et ceux en interne, la valeur des macros SIGRTMIN est réajustée. C'est un argument supplémentaire de ne jamais utiliser les signaux 
par leur valeur mais en utilisant la constante symbolique SIGRTMIN.

\paragraph{} En ce qui concerne le signal 0, lorsque nous écrivons kill(pid, 0), la documentation nous dit qu'aucun signal n'est envoyé mais que les conditions d'erreurs sont vérifées.
Cela signifie que le processus visé ne recevra aucun signal mais que le destinateur peut recevoir une erreur. Cela a son utilité pour vérifier si un programme existe ou non.

\section{Cas concrets}

\paragraph{} Une fois les bases vues, passons à la pratique avec quelques exemples de cas concrets utilisant les signaux et ses propriétés. Nous verrons que le champ des possibilités
est vaste et qu'une solution ne sera pas toujours suffisante pour couvrir toutes les possibilités.

\subsection{Un read non bloquant}

\paragraph{} L'idée est simple, lire au clavier pendant une certaine durée. L'appel système read étant bloquant tant que rien n'est lu au clavier, nous aimerions que notre programme
puisse reprendre après une certaine durée. Cela peut être utile sur un serveur par exemple. Le serveur attend une réponse du client et s'il n'a pas répondu endéans les 5 secondes, 
une valeur par défaut sera utilisée.

\paragraph{} Avant d'écrire le code, nous devons voir comment réagissent les appels systèmes lents tel que read lorsque le programme reçoit un signal. En effet, lorsqu'un signal est 
reçu et que l'ordonnanceur donne la main à notre processus, l'appel système est intérrompu et le gestionnaire de signal est appelé. Une fois le signal traité, les appels systèmes
lents nous disent que ceux-ci redémarrent si rien ne s'est passé. Hors, ce n'est pas ce que nous recherchons, nous ne voulons pas relire au clavier après que notre signal ait été
envoyé et traité. Pour cela, il est possible de modifier ce comportement gràce à la fonction \textit{siginterrupt} qui nous permet de préciser si l'appel système intérrompu doit ou non reprendre.

\paragraph{} Ecrivons maintenant un programme simple qui lit un caractère au clavier pendant 5 secondes. Si le caractère ENTER a été lu, un message disant "Gagné" sera envoyé.
Si rien n'a été lu avant ces 5 secondes ou que ce n'est pas bon le bon caractère, un message disant "Perdu" sera envoyé. Ici nous utiliserons la fonction alarm(5) qui envoie 
un signal SIGALRM à notre programme.

\lstinputlisting[style = code]{codes/read_time_out.c}

\paragraph{} Nous pourrions penser qu'aux premiers abords, ce programme n'ait pas de bugs. Néanmoins il existe un cas où l'ordonnanceur pourrait créer quelques problèmes.
Imaginons le cas où notre programme arrive à la ligne 30 et exécute l'instruction alarm(5) qui a pour but de lancer la minuterie. 
Ensuite l'ordonnanceur donne la main à un autre programme et, pour une raison ou l'autre, le système est fort chargé. Comme le système est chargé, notre programme ne reprend 
qu'après la sixième seconde. Lorsqu'il a à nouveau la main, le signal SIGALRM a déjà été envoyé car 5 secondes s'étaient écoulées donc le traitement doit d'abord être éxécuté.
Une fois, le traitement du signal fait, notre code peut continuer et éxécuter l'instruction suivante read(). Sauf que la minuterie étant déjà passée, notre processus sera bloqué 
par l'appel système et il ne recevra jamais le signal SIGALRM.

\paragraph{} Pour pallier à ce problème, nous devons rajouter quelques instructions supplémentaires afin d'éviter de faire un read si la minuterie est déjà passée.
Pour cela, nous allons utiliser les fonctions sigsetjmp et siglongjmp contenue dans le fichier d'en-tête $<setjmp.h>$. 

\paragraph{} Un premier appel à sigsetjmp enregistre l'environnement dans une variable et retourne 0. Ensuite, un appel à siglongjmp en donnant l'environnement donné par sigsetjmp
nous remet à l'instruction sigsetjmp et l'éxécute à nouveau mais retourne cette fois-ci une valeur différente de 0. Cela nous permet donc de revenir en arrière 
mais de ne pas forcément refaire les mêmes instructions car nous pouvons utiliser une condition afin d'éviter cela.

\paragraph{} Modifions un peu le code pour régler le problème.

\lstinputlisting[style = code]{codes/read_time_out2.c}

\paragraph{} Le comportement classique reste le même, notre read sera bien intérrompu par un signal ou par un caractère lu. Néanmoins, si l'alarme se déclenche après notre read, 
notre programme ne lancera pas le read grâce à sigsetjmp.

\paragraph{} Nous avons vu que les signaux pouvaient vite complexifier de simples codes. Voyons un autre cas qui peut poser quelques problèmes lorsque nous ne faisons pas attention.

\subsection{Gérer une zone critique}

\paragraph{} Lorsque nous utilisons un gestionnaire de signal, nous devons faire attention aux variables que nous manipulons. En effet, le traitement d'un signal utilise le 
même contexte que notre main(). C'est pour cela qu'il est déconseillé d'utiliser des fonctions utilisant des variables statiques ou globales dans un gestionnaire car il pourrait y avoir 
des conflits. Par exemple, si notre processus fait appel à malloc qui utilise des variables statiques, que notre code est intérrompu lors de l'appel à malloc, et que notre traitement
du signal fait lui aussi un malloc. Les variables statiques seront modifiées par le malloc du traitement du signal et le second malloc du code qui reprendra où il s'est arrêté 
aura des données modifiées qui seront incohérentes. Un autre exemple que nous allons illustrer est la gestion d'une zone critique avec une variable globale.

\paragraph{} Voici un simple exemple qui utilise une variable globale comme diviseur d'un nombre. La gestion du signal a pour but de modifier ce diviseur et de le mettre à 0.

\subsubsection*{Code}

\lstinputlisting[style = code]{codes/blocage_signal_sans_masque.c}

\subsubsection*{Output}

\lstinputlisting[style = output]{outputs/blocage_signal_sans_masque.txt}

\paragraph{} Lorsque nous lançons ce programme, nous pouvons voir que le traitement du signal modifie la valeur à 0 et provoque une division par zéro. Une façon de résoudre
ce problème est de protéger cette zone critique du programme et de bloquer l'arrivée de signaux. Pour cela, nous allons utiliser les masques de signaux.

\paragraph{} Les masques de signaux sont des masques qui contiennent un groupe de signal. Grâce à la fonction sigprocmask, nous pouvons bloquer ou débloquer l'arrivée des signaux
définis par le masque donné en paramètres. La structure représentant le masque est opaque. Cela signifie qu'il ne faut pas remplir les différents champs de la structure soi-même 
mais utiliser les fonctions mises à disposition. La foncction \textit{sigemptyset} permet d'initialiser le masque reçu en paramètre. Les fonctions \textit{sigaddset} et \textit{sigdelset} permet d'ajouter des signaux aux masques.
Toute tentative d'ajouter SIGKILL ou SIGSTOP sera tout simplement ignorée silencieusement par le système. 
Il existe d'autres fonctions associées aux masques que nous pouvons trouver à la page de manuel sigemptyset(3).

\paragraph{} Récrivons notre programme en bloquant l'arrivée des signaux lors de l'entrée en zone critique.

\subsubsection*{Code}

\lstinputlisting[style = code]{codes/blocage_signal_avec_masque.c}

\subsubsection*{Output}

\lstinputlisting[style = output]{outputs/blocage_signal_avec_masque.txt}

\paragraph{} Nous constatons que le signal est bien envoyé pendant la zone critique mais que celui-ci est mis en attente. Le signal a été traité lorsque le signal était débloqué.

\section{Limites et contraintes des signaux}

\subsection{Une bascule}

\paragraph{} Un dernier cas concret que nous allons réaliser est une bascule qui reçoit deux signaux différents. En fonction du signal reçu, la bascule se mettra à jour. 
La bascule a deux états possibles : UP et DOWN. Nous pourrions être tentés d'écrire un programme qui se met à jour en utilisant les 2 signaux classiques SIGUSR1 SIGUSR2 mis à la 
disposition du programmeur. Mais ce programme serait bien trop incomplet pour plusieurs raisons.
Tout d'abord les signaux classiques n'ont pas de file d'attente. Si notre programme reçoit trop de signaux d'un coup, l'état de notre bascule à un moment T ne serait pas forcément 
cohérent par rapport à la liste de signaux envoyés car les signaux accumulés seraient tout simplement perdus.

\subsubsection*{Exemple de code}

\lstinputlisting[style = code]{codes/blocage_signal_avec_masque.c}

\subsubsection*{Output}

\lstinputlisting[style = output]{outputs/file_attente_signal.txt}

\paragraph{} Nous pouvons observer que le premier signal est bloqué par le masque, mis dans la file d'attente et ensuite traité une fois masque levé. Les deux autres signaux envoyés
n'ont jamais été traités et ont été perdus.

\section{Les signaux temps-réel}

\subsection{Caractéristiques des signaux temps-réel}

\paragraph{} Pour écrire notre bascule, nous allons utiliser les signaux temps-réel qui, en plus d'ajouter des signaux supplémentaires pour le programmeur, ajoute une file d'attente
des signaux reçus. Une autre caractéristique qui les différencie de ceux classiques des signaux temps-réel sont leur priorité. La documentation ne précise pas de priorité de traitement
lorsque plusieurs signaux arrivent alors que les signaux temps-réel sont traités par odre croissant. Cela signifie que les signaux temps-réel de plus petite valeur seront traités 
en priorité. Enfin, les signaux temps-réel possèdent une structure supplémentaire permettant de stocker une variable entière ou un pointeur. Cela peut être utilisé pour communiquer 
des informations entre deux processus.

\subsection{Cas concret: une bascule}

\paragraph{} Les signaux temps-réel apporte donc tout ce dont nous avions besoin pour parvenir à coder cette bascule. Si nous gardons la même logique que nous avions imaginée plus tôt
en utilisant deux signaux différents : un premier pour l'état UP et un second pour l'état DOWN. Cela donnerait un code ressemblant à ceci.

\subsubsection*{Code}

\lstinputlisting[style = code]{codes/bascule.c}

\subsubsection*{Output}

\lstinputlisting[style = output]{outputs/bascule.txt}

\subsubsection*{Remarques}

\paragraph{} Nous bloquons l'arrivée des signaux afin de forcer notre bascule à gérer d'un coup une quantité de signaux d'un coup et d'illustrer la file d'attente. Nous pouvons
donc observer que les signaux sont bien mis dans l'ordre d'arrivée dans la file d'attente mais un problème persiste. Ils ne sont pas traités selon l'ordre d'arrivée mais selon
leur priorité. Cela est un problème car l'état de notre bascule ne correspond pas à l'arrivée des signaux. Pour y parvenir, nous devons donc utiliser un seul signal et pour pouvoir 
différencier un signal UP d'un signal DOWN, nous allons utiliser la dernière propriété de ces signaux en transmettant une valeur avec le signal.

\subsubsection*{Code}

\lstinputlisting[style = code]{codes/bascule_viable.c}

\subsubsection*{Output}

\lstinputlisting[style = output]{outputs/bascule_viable.txt}

\paragraph{} Nous voyons cette fois-ci que notre dernier problème est résolu et que notre bascule fonctionne correctement.

\section{Conclusion}

\paragraph{} Nous avons vu que les signaux classiques permettaient pas mal de possibilités mais que ceux-ci avaient leurs limites. Les signaux temps-réels les complètent mais 
modifient légèrement leur logique d'utilisation. Il faut aussi faire attention aux bugs que ceux-ci peuvent générer et savoir s'en prémunir. 
\end{document}
